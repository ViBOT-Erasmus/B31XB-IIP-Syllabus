%%%%%%%%%%%%%%%%%%%%%%%%%%%%%%%%%%%%% 
%% LE2I beamer template
%% Guillaume Lemaitre, October 2014
%%%%%%%%%%%%%%%%%%%%%%%%%%%%%%%%%%%%% 

\documentclass{beamer}

\usepackage[utf8]{inputenc}
\usepackage[T1]{fontenc} 
\usetheme{le2i} 

%% The amssymb package provides various useful mathematical symbols
\usepackage{amssymb}
%% The amsthm package provides extended theorem environments
\usepackage{amsthm}
%% amsmath for math environment
\usepackage{amsmath}

\DeclareMathOperator*{\argmin}{arg\,min}
\DeclareMathOperator*{\argmax}{arg\,max}
\DeclareMathOperator*{\sign}{sign}

%% figure package
\usepackage{epsf,graphicx}
\usepackage{epstopdf}
\usepackage{subfigure}
\usepackage{transparent}

%% In order to draw some graphs
\usepackage{tikz,xifthen}
\usepackage{tikz-qtree}
\usepackage{adjustbox}
\usetikzlibrary{decorations.pathmorphing}
\usetikzlibrary{fit}
\usetikzlibrary{backgrounds}
\usetikzlibrary{shapes,arrows,shadows}
\usetikzlibrary{calc,decorations.pathreplacing,decorations.markings,positioning}
\usetikzlibrary{snakes,decorations.text,shapes,patterns}
% \usepackage{scalefnt,lmodern,booktabs}

%% Package for cross and tick symbols
\usepackage{pifont}
\newcommand{\tick}{\color{green!60!black!80}\ding{51}}
\newcommand{\cross}{\color{red!60!black!80}\ding{55}}

\title{Introduction to Image Processing}
\author{Guillaume Lema\^itre \\ \texttt{guillaume.lemaitre@udg.edu}}
\date{Course syllabus \\ 16\textsuperscript{th} Sept. 2015}

\institute{Universit\'e de Bourgogne} 

%% Uncomment if you want to avoid thousand of bullet inside the menu
% \usepackage{etoolbox}
% \makeatletter
% \patchcmd{\slideentry}{\advance\beamer@xpos by1\relax}{}{}{}
% \def\beamer@subsectionentry#1#2#3#4#5{\advance\beamer@xpos by1\relax}%
% \makeatother

\begin{document}

% Show the title page
\begin{frame}
  \titlepage
\end{frame}

% Show the table of contents
\begin{frame}
  \tableofcontents[sectionstyle=show,subsectionstyle=show,subsubsectionstyle=hide]
\end{frame}

\section{Syllabus}

\subsection{Important Stuff}

\begin{frame}
  \frametitle{Important Stuff}
  % \framesubtitle{A Kick-Ass Title Subtitle}
  \begin{block}{Assessment}
    \begin{itemize}
    \item Mid-semester exam at the beginning of November (30 \%)
    \item Final exam at the end of the module (50 \%)
    \item Project with viva (20 \%)
    \end{itemize}
  \end{block}
\end{frame}

\subsection{Approximate Schedule}

\begin{frame}
  \frametitle{Approximate Schedule}
  % \framesubtitle{A Kick-Ass Title Subtitle}
  \only<1>{\begin{block}{Topic --- Load}
      \begin{enumerate}
      \item Digital image fundamentals (2h)
      \item Image enhancement in the spatial domain (3h)
      \item Spatial filtering (3h)
      \item Filtering in the frequency domain (3h)
      \item Restoration (2h)
      \item Introduction to the wavelet analysis (2h)
      \item Color space (2h)
      \item Mathematical morphology (2h)
      \item Image segmentation (6h)
      \end{enumerate}
    \end{block}}
  
  \only<2>{\begin{block}{Additionally ...}
      \begin{itemize}
      \item 2 practises of 2h
      \item 2 practises of 3h
      \item Project in team
      \end{itemize}
    \end{block}}
\end{frame}

\subsection{Textbooks}

\begin{frame}
  \frametitle{Approximate Schedule}
  % \framesubtitle{A Kick-Ass Title Subtitle}
  \begin{block}{Topic --- Load}
    \begin{itemize}
    \item \textbf{R. Gonzales \& R. Woods}, ``Digital Image Processing''
    \item \textbf{J. S. Lim}, ``Two Dimensional Signal and Image Processing''
    \item \textbf{J. Russ}, ``The Image Processing Handbook''
    \item \textbf{W. K. Pratt}, ``Digital Image Processing''
    \end{itemize}
  \end{block}
\end{frame}

\subsection{Prerequisites}

\begin{frame}
  \frametitle{Prerequisites}
  % \framesubtitle{A Kick-Ass Title Subtitle}
  \begin{block}{Requirements}
    \begin{columns}
      \begin{column}{.75\linewidth}
        \begin{itemize}
        \item<1-> Do you have a notebook?
        \item<2-> Do you have a linux distribution installed?
        \item<3->[$\rightarrow$] Install one!!! \texttt{http://www.ubuntu.com}
        \item<4-> Did you ever use python?
        \item<5->[$\rightarrow$] Install it!!! \textit{http://continuum.io/downloads}
        \item<6-> Did you ever use GitHub?
        \item<7->[$\rightarrow$] Start now!!! \textit{https://github.com}
        \item<7-> \textit{https://github.com/ViBOT-Erasmus/B31XB-IIP-Syllabus}
        \end{itemize}
      \end{column}
      \begin{column}{.3\linewidth}
        \onslide<3->{\begin{figure}
            \centering
            \includegraphics[width=1.\textwidth]{./images/ubuntu.png}
          \end{figure}}
        \onslide<5->{\begin{figure}
            \centering
            \includegraphics[width=.7\textwidth]{./images/anaconda.png}
          \end{figure}}
      \end{column}
    \end{columns}    
  \end{block}
\end{frame}

\section{Introduction to DIP}

\subsection{What is a digital image?}

\begin{frame}
  \frametitle{Introduction to DIP}
  \framesubtitle{What is a digital image?}
  \begin{block}{Type of digital images}
    \begin{columns}
      \begin{column}{.4\linewidth}
        \begin{itemize}
        \item<1-> Photographs
        \item<2-> Sonars
        \item<3-> Infrared cameras
        \item<4-> Ultrasound
        \item<5-> X-rays
        \item<6-> Magnetic resonance imaging
        \end{itemize}
      \end{column}
      \begin{column}{.6\linewidth}
        \only<1>{\begin{figure}
            \centering
            \includegraphics[width=.9\textwidth]{./images/intro/photo.jpg}
          \end{figure}}
        \only<2>{\begin{figure}
            \centering
            \includegraphics[width=.9\textwidth]{./images/intro/sonar.png}
          \end{figure}}
        \only<3>{\begin{figure}
            \centering
            \includegraphics[width=.9\textwidth]{./images/intro/ir.jpg}
          \end{figure}}
        \only<4>{\begin{figure}
            \centering
            \includegraphics[width=.9\textwidth]{./images/intro/us.jpg}
          \end{figure}}
        \only<5>{\begin{figure}
            \centering
            \includegraphics[width=.7\textwidth]{./images/intro/xray.jpg}
          \end{figure}}
        \only<6>{\begin{figure}
            \centering
            \includegraphics[width=.7\textwidth]{./images/intro/mri.jpg}
          \end{figure}}
      \end{column}
    \end{columns}    
  \end{block}
\end{frame}

\subsection{Application examples}

\begin{frame}
  \frametitle{Introduction to DIP}
  % \framesubtitle{Application examples}
  \begin{block}{Applications}
    \begin{columns}
      \begin{column}{.4\linewidth}
        \begin{itemize}
        \item<1-> Biomedical
        \item<2-> Cartography
        \item<3-> Land monitoring
        \item<4-> Inspection
        \item<5-> Surveillance
        \item<6-> Biometrics
        \item<7-> Enhanced visualisation
        \item<8-> Content-based image retrieval 
        \end{itemize}
      \end{column}
      \begin{column}{.6\linewidth}
        \only<1>{\begin{figure}
            \centering
            \includegraphics[width=.9\textwidth]{./images/intro/brain.jpg}
          \end{figure}}
        \only<2>{\begin{figure}
            \centering
            \includegraphics[width=.9\textwidth]{./images/intro/cart.jpeg}
          \end{figure}}
        \only<3>{\begin{figure}
            \centering
            \includegraphics[width=.9\textwidth]{./images/intro/lidar.jpg}
          \end{figure}}
        \only<4>{\begin{figure}
            \centering
            \includegraphics[width=.9\textwidth]{./images/intro/beer.jpg}
          \end{figure}}
        \only<5>{\begin{figure}
            \centering
            \includegraphics[width=.9\textwidth]{./images/intro/surveillance.jpg}
          \end{figure}}
        \only<6>{\begin{figure}
            \centering
            \includegraphics[width=.9\textwidth]{./images/intro/csi.jpg}
          \end{figure}}
        \only<7>{\begin{figure}
            \centering
            \includegraphics[width=.9\textwidth]{./images/intro/enhancement.jpg}
          \end{figure}}
        \only<8>{\begin{figure}
            \centering
            \includegraphics[width=.9\textwidth]{./images/intro/retrieval.jpg}
          \end{figure}}
      \end{column}
    \end{columns}    
  \end{block}
\end{frame}

\end{document}